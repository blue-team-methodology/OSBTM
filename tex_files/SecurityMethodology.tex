\section{Security Methodology}

\subsubsection{Classification}

We need to classify data by thinking about two main concepts:

\begin{enumerate}
    \item Data value
    \item Data confidential level. In this case, a classification like used by intelligence agencies is a good starting point:
    \begin{enumerate}[label=\alph*.]
        \item For your eyes only
        \item Top Secret
        \item Secret
        \item Confidential
        \item Unclassified
    \end{enumerate}
\end{enumerate}

Each company should define the sensitivity level of their data. 
PII should be protected very well. 
Based on the then created topology we can put the right building block to protect the asset in place. 

\subsubsection{Separation}

After classifying the data we can think about separation. 
Design the network based on the new categories and implement risk measures based on your sensitivity level. 
Companies often avoid separating data due to concerns about rising hardware, license, and management costs. 
However, these costs can be strategically managed.

\begin{itemize}[leftmargin=*]
    \item \textbf{Reserve physical separation} for only the most critical and sensitive data.
    \item \textbf{Use software-based separation} (e.g., VLANs, SDN, Virtualization) to achieve effective isolation at a lower cost.
    \item \textbf{Reduce licensing costs} by considering enterprise-grade open-source alternatives
    \item \textbf{Control management costs} by standardizing on a limited number of system types, which simplifies maintenance.
\end{itemize}

Separation is a critical security practice for two key reasons:

\begin{enumerate}
    \item \textbf{It Prevents Privilege Escalation:} If an attacker gains elevated permissions on a single, aggregated system, they can potentially access all the data stored on it. Separation contains the blast radius of such an attack.
    \item \textbf{It Creates Security Barriers:} Forcing an intruder to navigate multiple, separate systems creates layers of defense that make it significantly harder to access different levels of data.
\end{enumerate}

\noindent The principle of separation must include administrative roles. 
Granting a single ``Domain Admin'' account universal access is a major security risk. 
Different systems should have different administrators to enforce a separation of duties. 
If a master account is compromised, the entire network is exposed.

\subsubsection{Secure Logging}

Effective security monitoring requires comprehensive data. 
Configure all network devices, servers, and applications to generate logs. 
Go beyond the defaults by implementing custom audit systems to track the most sensitive operations. 
Logs should never be left on the source device. 
Attackers routinely delete local logs to hide their activity after a compromise. 
To prevent this, all log data must be sent in real-time to a central log server.
This creates a secure, off-site record of all activity that cannot be erased by an intruder.
The central log server must be a protected. 
Its administration should be separate from the standard IT department. 
A "Domain Admin" should not have the ability to modify or delete logs. 
This separation of duties ensures that no single individual with stolen credentials can breach your systems and erase the evidence.

\subsubsection{Services}
Services are the channels that users (employees, customers) utilize to access data. 
They are the means, not the end goal, of an information system. 
However, every service that is enabled adds a layer of risk and complexity. 
Each service is both a potential attack vector and a pathway to bypass established data classification. 
Therefore, when designing data access policies, the KISS ("Keep It Simple, Stupid") principle is a critical security principle. 
The more methods you provide for users to access data, the less secure the overall system becomes.

Common Points of Data Security Failure:
\begin{itemize}
    \item Physical Media: Users download data from a secure server onto a personal thumb drive.
    \item Cloud Services: Users send corporate data from a secure server to a public cloud platform.
    \item Email Communication: Users attach sensitive files from a secure repository to an email.
    \item Web Applications: Users access and potentially exfiltrate data from a secure backend via a web interface.
    \item Data Export: Users query a database and publish the results in an unsecured document.
\end{itemize}

When data moves from one service to another, its classification and security controls must follow it. 
If the classification is lost in transit, the security model is fundamentally broken.

\subsubsection{Inventory}
A comprehensive and strictly controlled inventory of all assets is important. 
Without adequate controls, it is far too easy for unauthorized devices to be brought into a company and connected to the internal Wi-Fi and corporate network. 
In many cases, such devices go undiscovered for months, creating significant and persistent security vulnerabilities.
This inventory must be exhaustive, encompassing not only hardware but also all users, groups, active services, data classification levels, operating systems, software patch levels, access permissions, and audit logs. 
As a guiding principle, any device or entity not explicitly recorded in the inventory must be prevented from connecting to the company's information system.

\subsubsection{Identification}
Everything in a modern information system needs a digital identity—not just users and guests, but devices and services as well. 
Usernames and passwords are a thing of the past; multi-factor and biometric authentication should be the standard. 
It's important to remember that today's technology still cannot perfectly identify a human being through digital means alone. 
So, while you should enhance your authentication systems, never assume they are foolproof. 
It is always better to have a system that sometimes rejects a valid user (a false negative) than one that might accept an invalid user (a false positive).

\subsubsection{Trust no one (X-Files)}

\textit{The Zero Trust Principle}: In modern cybersecurity, trust is a vulnerability. 
The foundational principle is to never trust, always verify. 
Consequently, the internal network should not be considered a secure place. 
Every machine, user, and service account should be treated as potentially malicious, and IP addresses should never be a basis for trust. 
This approach requires treating your entire network as a hostile environment, securing each device as if it were directly exposed to the public internet.\\

\textit{Data Security in Transit}: Critical Data must always be encrypted in transit and carry a persistent classification label. 
This label must be checked at every security checkpoint, and the data's classification level should be strictly maintained, never elevated or declassified while moving. 
This principle is formalized in security frameworks like the Bell-LaPadula model.
\textit{Bidirectional Traffic Inspection:}
This security posture is especially critical for firewall configuration. 
A common and dangerous mistake is to block inbound traffic while implicitly trusting all outbound traffic. 
Instead, you must rigorously inspect data flow in both directions. 
Treat both the internal network and the internet as untrusted, checking all data as it moves between any two network segments, even those inside your own system.\
